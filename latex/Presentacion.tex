\documentclass{beamer}
\usetheme{Madrid}
\usecolortheme{dolphin}

\usepackage[utf8]{inputenc}
\usepackage[spanish]{babel}
\usepackage{listings}
\usepackage{graphicx}
\usepackage{xcolor}

% Configuración de listings para código Python
\definecolor{codegreen}{rgb}{0,0.6,0}
\definecolor{codegray}{rgb}{0.5,0.5,0.5}
\definecolor{codepurple}{rgb}{0.58,0,0.82}
\definecolor{backcolour}{rgb}{0.95,0.95,0.92}

\lstdefinestyle{mystyle}{
    backgroundcolor=\color{backcolour},   
    commentstyle=\color{codegreen},
    keywordstyle=\color{magenta},
    numberstyle=\tiny\color{codegray},
    stringstyle=\color{codepurple},
    basicstyle=\ttfamily\footnotesize,
    breakatwhitespace=false,         
    breaklines=true,                 
    captionpos=b,                    
    keepspaces=true,                 
    numbers=left,                    
    numbersep=5pt,                  
    showspaces=false,                
    showstringspaces=false,
    showtabs=false,                  
    tabsize=2
}

\lstset{style=mystyle, language=Python}

\title{Sistema de Gestión de Minimarket}
\subtitle{Proyecto de Programación Orientada a Objetos}
\author{Martin Rodriguez \and Juan Aguilar \and Ignacio Iturra \and Diego Dominguez}
\institute{Universidad de Los Lagos}
\date{\today}

\begin{document}

% Diapositiva de Título
\begin{frame}
    \titlepage
\end{frame}

% Índice
\begin{frame}{Contenido}
    \tableofcontents
\end{frame}

% Sección 1: Introducción
\section{Introducción}
\begin{frame}{Introducción}
    \begin{itemize}
        \item \textbf{Contexto:} Los pequeños negocios (minimarkets) a menudo dependen de procesos manuales para el inventario y las ventas.
        \item \textbf{Problema:} 
            \begin{itemize}
                \item Errores en el cálculo de stock.
                \item Pérdidas financieras desconocidas.
                \item Lentitud en la atención al cliente.
            \end{itemize}
        \item \textbf{Solución:} Un sistema de escritorio en Python que automatiza la gestión de inventario y ventas, mejorando la eficiencia y el control.
    \end{itemize}
\end{frame}

% Sección 2: Objetivos
\section{Objetivos}
\begin{frame}{Objetivos del Proyecto}
    \textbf{Objetivo General:}
    \begin{itemize}
        \item Desarrollar un sistema de información para administrar un minimarket, gestionando inventario y ventas de forma eficiente.
    \end{itemize}
    
    \vspace{0.5cm}
    
    \textbf{Objetivos Específicos:}
    \begin{itemize}
        \item Implementar arquitectura \textbf{MVC} (Modelo-Vista-Controlador).
        \item Diseñar una \textbf{GUI} funcional con Tkinter.
        \item Gestionar productos con \textbf{Categorías} y \textbf{Unidades}.
        \item Implementar persistencia de datos con archivos \textbf{JSON}.
        \item Generar reportes de ventas con IDs únicos.
    \end{itemize}
\end{frame}

% Sección 3: Fundamentación Técnica
\section{Fundamentación Técnica}
\begin{frame}{Tecnologías y Patrones}
    \begin{columns}
        \column{0.5\textwidth}
        \textbf{Tecnologías:}
        \begin{itemize}
            \item \textbf{Python 3.14:} Lenguaje principal.
            \item \textbf{Tkinter:} Interfaz gráfica.
            \item \textbf{JSON:} Persistencia de datos ligera.
        \end{itemize}
        
        \column{0.5\textwidth}
        \textbf{Patrones de Diseño:}
        \begin{itemize}
            \item \textbf{MVC:} Separación de lógica (Controlador), datos (Modelo) e interfaz (Vista).
            \item \textbf{Facade:} \texttt{SupermercadoController} unifica el acceso a los sub-controladores.
            \item \textbf{POO:} Uso de Clases, Herencia y Composición.
        \end{itemize}
    \end{columns}
\end{frame}

% Sección 4: Arquitectura
\section{Arquitectura del Sistema}
\begin{frame}{Diagrama de Clases}
    \begin{figure}
        \centering
        \includegraphics[width=\textwidth, height=0.75\textheight, keepaspectratio]{images/diagrama_clases.png}
        \caption{Estructura de Clases del Sistema (MVC)}
    \end{figure}
\end{frame}

\begin{frame}{Estructura del Proyecto}
    La estructura de carpetas refleja la arquitectura MVC:
    \vspace{0.5cm}
    \begin{itemize}
        \item \texttt{models/}: Definición de objetos (\texttt{Producto}, \texttt{Venta}, \texttt{Usuario}).
        \item \texttt{views/}: Interfaz gráfica (\texttt{SupermercadoGUI}).
        \item \texttt{controllers/}: Lógica de negocio (\texttt{VentaController}, \texttt{ProductoController}).
        \item \texttt{data/}: Archivos de persistencia (\texttt{.json}).
    \end{itemize}
\end{frame}

% Sección 5: Implementación (Código)
\section{Implementación y Algoritmos}

\begin{frame}[fragile]{Modelo: Producto (Composición)}
    Uso de composición para \texttt{Categoria} y \texttt{Unidad}.
    \begin{lstlisting}
class Producto:
    def __init__(self, codigo, nombre, precio, stock, 
                 categoria: Categoria, unidad: Unidad, ...):
        self.codigo = codigo
        self.nombre = nombre
        self.precio = precio
        self.stock = stock
        self.categoria = categoria # Objeto Categoria
        self.unidad = unidad       # Objeto Unidad
    
    def to_dict(self) -> dict:
        return {
            'codigo': self.codigo,
            # ...
            'categoria': self.categoria.nombre,
            'unidad': self.unidad.nombre
        }
    \end{lstlisting}
\end{frame}

\begin{frame}[fragile]{Controlador: Generación de IDs}
    Algoritmo para generar IDs autoincrementales en \texttt{VentaController}.
    \begin{lstlisting}
def obtener_siguiente_id(self) -> int:
    """Calcula el siguiente ID basado en el historial."""
    if not self.ventas:
        return 1
    max_id = 0
    for v in self.ventas:
        v_id = v.get('id')
        if isinstance(v_id, int) and v_id > max_id:
            max_id = v_id
    return max_id + 1
    \end{lstlisting}
\end{frame}

\begin{frame}[fragile]{Controlador: Realizar Venta}
    Lógica transaccional: Crear venta, descontar stock y guardar.
    \begin{lstlisting}
def realizar_venta(self, items: List[tuple]) -> Optional[Venta]:
    nuevo_id = self.obtener_siguiente_id()
    venta = Venta(id_venta=nuevo_id)
    
    for codigo, cantidad in items:
        producto = self.producto_controller.productos[codigo]
        venta.agregar_item(producto, cantidad)
        # Actualizacion de stock en tiempo real
        self.producto_controller.actualizar_stock(
            codigo, cantidad, 'restar'
        )
    
    self.ventas.append(venta.to_dict())
    self.guardar_ventas()
    return venta
    \end{lstlisting}
\end{frame}

% Sección 6: Funcionalidades
\section{Funcionalidades Clave}
\begin{frame}{Funcionalidades del Sistema}
    \begin{enumerate}
        \item \textbf{Control de Acceso:} Login con roles (Admin/Comprador).
        \item \textbf{Gestión de Inventario:} 
            \begin{itemize}
                \item CRUD de Productos.
                \item Alertas de stock bajo.
            \end{itemize}
        \item \textbf{Punto de Venta (POS):}
            \begin{itemize}
                \item Búsqueda rápida de productos.
                \item Cálculo automático de totales.
                \item Generación de boleta (ID único).
            \end{itemize}
        \item \textbf{Administración:}
            \begin{itemize}
                \item Creación de nuevos administradores.
                \item Reportes de ventas detallados.
            \end{itemize}
    \end{enumerate}
\end{frame}

% Sección 7: Conclusión
\section{Conclusión}
\begin{frame}{Conclusiones y Trabajo Futuro}
    \textbf{Conclusiones:}
    \begin{itemize}
        \item Se logró un sistema robusto y modular gracias a MVC.
        \item La separación de controladores facilita el mantenimiento.
        \item La interfaz gráfica es intuitiva para el usuario final.
    \end{itemize}
    
    \vspace{0.5cm}
    
    \textbf{Trabajo Futuro:}
    \begin{itemize}
        \item Migración de JSON a Base de Datos SQL (SQLite/PostgreSQL).
        \item Implementación de reportes gráficos (matplotlib).
        \item Soporte para lector de código de barras.
    \end{itemize}
\end{frame}

\begin{frame}
    \centering
    \Huge \textbf{¡Muchas Gracias!}
    
    \vspace{1cm}
    \large ¿Preguntas?
\end{frame}

\end{document}
