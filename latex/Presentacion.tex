\documentclass{beamer}
\usetheme{Madrid}
\usecolortheme{dolphin}

\usepackage[utf8]{inputenc}
\usepackage[spanish]{babel}
\usepackage{listings}
\usepackage{graphicx}
\usepackage{xcolor}
\usepackage{booktabs}

% Configuración de listings para código Python
\definecolor{codegreen}{rgb}{0,0.6,0}
\definecolor{codegray}{rgb}{0.5,0.5,0.5}
\definecolor{codepurple}{rgb}{0.58,0,0.82}
\definecolor{backcolour}{rgb}{0.95,0.95,0.92}

\lstdefinestyle{mystyle}{
    backgroundcolor=\color{backcolour},   
    commentstyle=\color{codegreen},
    keywordstyle=\color{magenta},
    numberstyle=\tiny\color{codegray},
    stringstyle=\color{codepurple},
    basicstyle=\ttfamily\tiny,
    breakatwhitespace=false,         
    breaklines=true,                 
    captionpos=b,                    
    keepspaces=true,                 
    numbers=left,                    
    numbersep=5pt,                  
    showspaces=false,                
    showstringspaces=false,
    showtabs=false,                  
    tabsize=2
}

\lstset{style=mystyle, language=Python}

\title{Sistema de administración}
\subtitle{Funcionalidades}
\author{Martin Rodriguez}
\institute{Programación Orientada a Objetos}
\date{\today}

\begin{document}

% Diapositiva de Título
\begin{frame}
    \titlepage
\end{frame}

% Sección: Diagrama de Clases
\section{Diagrama de Clases}
\begin{frame}{Diagrama de Clases}
    \begin{figure}
        \centering
        \includegraphics[width=\textwidth, height=0.8\textheight, keepaspectratio]{images/diagrama_clases.png}
        \caption{Estructura de Clases del Sistema (MVC)}
    \end{figure}
\end{frame}

% Sección: Algoritmos Actualizados
\section{Implementación y Algoritmos}

\begin{frame}[fragile]{Modelo: Producto (Soporte Multimedia)}
    La clase \texttt{Producto} ahora integra rutas de imágenes y objetos de Unidad.
    \begin{lstlisting}
class Producto:
    # Constructor de la clase Producto
    def __init__(self, codigo, nombre, precio, stock, 
                 categoria, unidad, stock_minimo=5, imagen_path=None):
        self.codigo = codigo # Identificador unico del producto
        self.nombre = nombre # Nombre descriptivo del producto
        self.precio = precio # Precio unitario de venta
        self.stock = stock # Cantidad disponible en inventario
        self.categoria = categoria # Objeto Categoria asociado
        self.unidad = unidad       # Objeto Unidad (Kg/Unid)
        self.imagen_path = imagen_path # Ruta local del archivo de imagen
        self.stock_minimo = stock_minimo # Umbral para alertas de stock bajo

    # Metodo para verificar si el stock es critico
    def tiene_stock_bajo(self) -> bool:
        return self.stock <= self.stock_minimo # Retorna True si stock <= minimo
    \end{lstlisting}
\end{frame}

\begin{frame}[fragile]{Vista: Carga de Imágenes (Pillow)}
    Algoritmo en \texttt{SupermercadoGUI} para renderizar imágenes en ventanas emergentes.
    \begin{lstlisting}
from PIL import Image, ImageTk # Importacion de libreria de imagenes

# Metodo para mostrar detalles visuales del producto
def mostrar_detalle_producto(self, event):
    # ... obtener producto ...
    # Verifica si existe una ruta de imagen valida
    if producto.imagen_path and os.path.exists(producto.imagen_path):
        try:
            img = Image.open(producto.imagen_path) # Abre el archivo de imagen
            # Redimensionamiento con filtro LANCZOS para alta calidad
            img.thumbnail((300, 300), Image.Resampling.LANCZOS)
            photo = ImageTk.PhotoImage(img) # Convierte a formato compatible con Tkinter
            lbl_img = ttk.Label(detalle, image=photo) # Crea widget de etiqueta con imagen
            lbl_img.image = photo # Mantiene referencia para evitar Garbage Collection
            lbl_img.pack() # Muestra la imagen en la interfaz
        except Exception as e:
            print(f"Error imagen: {e}") # Manejo de errores de carga
    \end{lstlisting}
\end{frame}

\begin{frame}[fragile]{Controlador: Actualización de Stock (Polimorfismo de Datos)}
    Manejo de tipos de datos según la unidad (float para Kg, int para Unidades).
    \begin{lstlisting}
# Metodo para modificar el inventario
def actualizar_stock(self, codigo: str, cantidad: float, operacion: str):
    producto = self.productos.get(codigo) # Busca el producto por codigo
    
    if operacion == 'restar': # Logica para venta o reduccion
        if producto.stock < cantidad: # Validacion de stock suficiente
            return False # Retorna falso si no hay suficiente stock
        producto.stock -= cantidad # Resta la cantidad del stock actual
    elif operacion == 'agregar': # Logica para reabastecimiento
        producto.stock += cantidad # Suma la cantidad al stock actual
        
    # Verificación de alertas post-operación
    if producto.tiene_stock_bajo(): # Chequea si se cruzo el umbral minimo
        print(f"ALERTA: Stock bajo en {producto.nombre}") # Notifica alerta
        
    self.guardar_productos() # Persiste los cambios en JSON
    return True # Retorna exito
    \end{lstlisting}
\end{frame}

% Sección: Funcionalidades Detalladas
\section{Funcionalidades Detalladas}

\begin{frame}{1. Acceso y Autenticación}
    \textbf{Pantalla de Login:}
    \begin{itemize}
        \item Campo de texto: \texttt{Usuario}.
        \item Campo de contraseña: \texttt{Contraseña} (oculta con *).
        \item Botón \textbf{Ingresar}: Valida credenciales y redirige según rol.
        \item Botón \textbf{Registrarse}: Abre formulario de registro.
    \end{itemize}
    
    \textbf{Pantalla de Registro:}
    \begin{itemize}
        \item Campos para nuevo usuario y contraseña.
        \item Botón \textbf{Registrar}: Crea cuenta tipo "Comprador".
        \item Botón \textbf{Volver}: Retorna al login.
    \end{itemize}
\end{frame}

\begin{frame}{2. Panel de Inventario (Administrador) - Parte 1}
    \textbf{Visualización y Búsqueda:}
    \begin{itemize}
        \item \textbf{Barra de Búsqueda:} Filtrado por nombre, código o categoría.
        \item \textbf{Interacción:} Doble click en fila para ver \textbf{Ficha Técnica con Imagen}.
    \end{itemize}
\end{frame}

\begin{frame}{2. Panel de Inventario (Administrador) - Parte 2}
    \textbf{Botones de Acción:}
    \begin{itemize}
        \item \textbf{Nuevo Producto:} Abre formulario para crear producto (Nombre, Precio, Stock, Categoría, Unidad, Stock Mínimo, Imagen).
        \item \textbf{Editar Producto:} Modifica todos los atributos de un producto existente.
        \item \textbf{Actualizar Stock:} Entrada rápida para sumar/restar cantidad al inventario.
        \item \textbf{Eliminar Producto:} Borrado lógico/físico del sistema.
        \item \textbf{Reiniciar Productos:} Restaura la base de datos a valores de fábrica.
        \item \textbf{Crear Admin:} Permite elevar privilegios de usuarios.
        \item \textbf{Exportar CSV:} Descarga el inventario actual a archivo Excel/CSV.
    \end{itemize}
\end{frame}

\begin{frame}{3. Punto de Venta}
    \textbf{Panel de Selección:}
    \begin{itemize}
        \item \textbf{Buscador/Scanner:} Campo para ingresar código o nombre.
        \item \textbf{Selector de Cantidad:} Campo numérico (acepta decimales para Kg).
        \item Botón \textbf{Agregar}: Mueve el ítem al carrito.
    \end{itemize}
    
    \textbf{Panel de Carrito:}
    \begin{itemize}
        \item Botón \textbf{Limpiar Carrito}: Cancela la operación actual.
        \item Botón \textbf{Confirmar Venta}: Procesa la transacción, descuenta stock y genera ID.
    \end{itemize}
\end{frame}

\begin{frame}{4. Reportes y Alertas}
    \textbf{Pestaña Reportes:}
    \begin{itemize}
        \item \textbf{Detalle de Venta:} Doble click muestra desglose de productos vendidos.
    \end{itemize}
    
    \textbf{Pestaña Alertas:}
    \begin{itemize}
        \item \textbf{Tabla de Riesgo:} Filtra automáticamente productos con stock $\le$ mínimo.
        \item Botón \textbf{Actualizar}: Refresca el listado de alertas.
    \end{itemize}
\end{frame}

\begin{frame}{5. Perfil Comprador}
    Funcionalidades restringidas para clientes:
    \begin{itemize}
        \item \textbf{Pestaña Comprar:} Interfaz POS simplificada para auto-atención.
        \item \textbf{Pestaña Catálogo:}
            \begin{itemize}
                \item Vista de solo lectura del inventario.
                \item Buscador habilitado.
                \item Visualización de imágenes (Doble click).
                \item Sin acceso a edición ni stock interno.
            \end{itemize}
    \end{itemize}
\end{frame}

\end{document}
