\documentclass{beamer}
\usetheme{Madrid}
\usecolortheme{default}

\title{Sistema de Gestión de Minimarket}
\subtitle{Proyecto de Programación Orientada a Objetos}
\author{Martin Rodriguez \and Juan Aguilar \and Ignacio Iturra \and Diego Dominguez}
\institute{Departamento de Ciencias de La Ingeniería \\ Ingeniería Civil en Informática}
\date{\today}

\begin{document}

% Diapositiva de Título
\begin{frame}
    \titlepage
\end{frame}

% Índice
\begin{frame}{Contenidos}
    \tableofcontents
\end{frame}

% 1. Descripción de la Problemática
\section{Descripción de la Problemática}
\begin{frame}{Descripción de la Problemática}
    \begin{itemize}
        \item \textbf{Gestión Manual:} Muchos minimarkets aún dependen de cuadernos o métodos manuales.
        \item \textbf{Errores de Inventario:} Dificultad para mantener un conteo exacto del stock real vs. registrado.
        \item \textbf{Lentitud en Ventas:} El cálculo manual de totales ralentiza la atención al cliente.
        \item \textbf{Falta de Información:} Ausencia de reportes claros sobre ganancias o productos más vendidos.
        \item \textbf{Pérdidas:} Mermas no detectadas y quiebres de stock inesperados.
    \end{itemize}
\end{frame}

% 2. Objetivos
\section{Objetivos}
\begin{frame}{Objetivos}
    \textbf{Objetivo General}
    \vspace{0.5em}
    
    Desarrollar un sistema de información de escritorio para la administración de un minimarket, que permita gestionar el inventario y registrar ventas de manera eficiente y segura.
    
    \vspace{1em}
    \textbf{Objetivos Específicos}
    \begin{itemize}
        \item Implementar arquitectura \textbf{MVC} con controladores segregados.
        \item Diseñar una \textbf{GUI} intuitiva con Tkinter.
        \item Desarrollar módulos de \textbf{Inventario} (con categorías y unidades) y \textbf{Ventas}.
        \item Implementar sistema de \textbf{Autenticación} y roles (Admin/Comprador).
        \item Asegurar persistencia de datos mediante archivos \textbf{JSON}.
        \item Generar \textbf{IDs únicos} para trazabilidad de ventas.
    \end{itemize}
\end{frame}

% 3. Justificación
\section{Justificación}
\begin{frame}{Justificación}
    \begin{block}{¿Por qué este sistema?}
        \begin{itemize}
            \item \textbf{Eficiencia Operativa:} Automatiza cálculos y actualizaciones de stock.
            \item \textbf{Integridad de Datos:} Reduce el error humano en el registro de transacciones.
            \item \textbf{Toma de Decisiones:} Provee reportes y alertas de stock crítico.
            \item \textbf{Escalabilidad:} Arquitectura modular (POO/MVC) que facilita futuras mejoras.
            \item \textbf{Accesibilidad:} Solución de bajo costo y fácil implementación (Python).
        \end{itemize}
    \end{block}
\end{frame}

% 4. Diagramas de Clases
\section{Diagramas de Clases}
\begin{frame}{Arquitectura del Sistema (MVC)}
    \textbf{Estructura Modular:}
    \vspace{1em}
    \begin{columns}
        \column{0.33\textwidth}
        \textbf{Modelos (Data)}
        \begin{itemize}
            \item Producto
            \item Venta
            \item Usuario
            \item Categoria
            \item Unidad
        \end{itemize}
        
        \column{0.33\textwidth}
        \textbf{Vistas (GUI)}
        \begin{itemize}
            \item SupermercadoGUI
            \item LoginWindow
            \item RegistroWindow
        \end{itemize}
        
        \column{0.33\textwidth}
        \textbf{Controladores (Logic)}
        \begin{itemize}
            \item ProductoController
            \item VentaController
            \item UsuarioController
            \item SupermercadoController (Fachada)
        \end{itemize}
    \end{columns}
    
    \vspace{1em}
    \textit{* Se recomienda insertar aquí la imagen del Diagrama de Clases UML generado.}
\end{frame}

% 5. Listado de Funcionalidades
\section{Funcionalidades}
\begin{frame}{Listado de Funcionalidades (1/2)}
    \begin{enumerate}
        \item \textbf{Autenticación y Roles:}
        \begin{itemize}
            \item Login seguro para Administradores y Compradores.
            \item Creación de nuevos administradores (solo por admins).
        \end{itemize}
        
        \item \textbf{Gestión de Inventario:}
        \begin{itemize}
            \item CRUD completo de productos.
            \item Soporte para Categorías y Unidades (Kilos/Unidades).
            \item Búsqueda en tiempo real.
        \end{itemize}
        
        \item \textbf{Control de Stock:}
        \begin{itemize}
            \item Actualización de stock (Agregar/Restar).
            \item Alertas automáticas de stock bajo.
        \end{itemize}
    \end{enumerate}
\end{frame}

\begin{frame}{Listado de Funcionalidades (2/2)}
    \begin{enumerate}
        \setcounter{enumi}{3}
        \item \textbf{Sistema de Ventas:}
        \begin{itemize}
            \item Carrito de compras interactivo.
            \item Validación de stock disponible antes de la venta.
            \item Generación de ID único por venta (Boleta).
        \end{itemize}
        
        \item \textbf{Reportes y Estadísticas:}
        \begin{itemize}
            \item Historial de últimas ventas con detalle.
            \item Resumen financiero (Ingresos, Valor Inventario).
        \end{itemize}
        
        \item \textbf{Persistencia y Seguridad:}
        \begin{itemize}
            \item Guardado automático en JSON separados.
            \item Opción de reinicio de fábrica (solo productos).
        \end{itemize}
    \end{enumerate}
\end{frame}

% Conclusión / Cierre
\begin{frame}
    \centering
    \Huge \textbf{¡Gracias por su atención!}
    
    \vspace{1cm}
    \Large ¿Preguntas?
\end{frame}

\end{document}
